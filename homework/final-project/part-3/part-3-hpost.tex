\documentclass[a4paper,12pt]{article}

\usepackage[utf8]{inputenc}
\usepackage{lipsum}  
\usepackage{url}
\usepackage[margin=1in]{geometry}

\usepackage{markdown}
\usepackage{setspace}
\usepackage{hyperref}
\usepackage{breakurl}

\title{HIST 380 Part 3: \\ Workplace equity differences with \\ respect to time, gender, and race }
\author{Henry Post, \url{hpost@iit.edu}}

\begin{document}

\maketitle

\newpage
\doublespacing

% 'In this paper' is a crappy way of starting. Boring! Change this!!!

	In this paper, I will examine the state of diversity and equality in the workplace in the 2010s onwards, and how that compares to diversity and equality in the 1940s through the 1980s. 
This paper will draw sources %don't like this
from studies that have both quantitative and qualitative content to take full advantage of all available information.

	The purpose of this comparison is to answer the question, ``How has workplace equality changed over time with respect to gender and race?'' Social norms have changed favorably since the 1940s for women and minorities, especially black people, but discrimination certainly still exists in many ways. Some are enforced by society as a whole through popular culture and racist beliefs, but others are enforced through socioeconomic power structures like through workplace discrimination, unfair hiring practices, concentrating wealth, school selection procedures, etc.

% from 2nd paper
% In fact, there was one black-owned company called Data Preparation Corp in Philidelphia, owned at the time by Julius A. (Jack) Baylor that dealt with data processing. This company was featured in an Ebony magazine article. The article details how Jack started with about \$20,000 and failed multiple times to start a company, but eventually succeeded and his company making about \$230,000 gross profit in the first two years.

% The first article I will talk about is a (black guy with the company in boston that hires black workers in 1979) \cite{csc}

% - shows USA had some good places that hired black people in late 1970s

% - is this just a coincedince? probably not.

% >> find another article talking about this guy!!!

% - does this show that britain had it a lot worse in terms of how beneficial its job market was to black people in STEM?

% >> It certainly suggests that (forgot my train of thought)

% >> what if there's just no sources though?

% >>> I also had trouble finding ANY articles like Ebony magazine in britain mags or even job postings for black people


	The first article I will talk about is one from Ebony Magazine Vol. 24, No. 6, Page 113\cite{punchcardpros}. This article is an advertisement for a company owned by Julius A. (Jack) Baylor, the CEO of Data Preparation Corp. This article was published on Dec. 1979.

	In the picture in the article, I am assuming most employees who perform data entry are women because of the haircuts and facial features that you can see in the article.

	This ad shows a company that has a black CEO and employs primarily black women. It touts the accuracy and speed of its workers as well as advertising to other black people that they should join the workforce. 
	
	This shows that there did exist black-owned companies with progressive hiring strategies, even in 1979. This is an example of a corporation that was successful, likely %is this a stretch? I /am/ biased.
	because it leveraged black women, who were relatively underutilized for high-paying jobs in the 1970s and 1980s. %I am assming this TODO back this up with sources!!!

	Jack mentions etc TODO TODO TODO TODO had failure before, multiple ones. I failed them just like their white employers. this is proof that white employers were known in cultural circles to fail black employees in various ways

	The next article\cite{pew} is a meta-analysis %is it tho?
by Pew Social Trends. This article is an aggregate of many other statistics and quantitative studies that were conducted over a broad range of ethnicities and genders. They draw their data from two primary sources: The 1990 and 2000 decennial censuses, and a nationally representative %what does this mean?
survey conducted from July 11th, 2017 to August 10th, 2017. 

% this whole section is excessive. Who cares about this info? How does it meet my end goal of comparing stuff? Was I wasting time here?

	The survey was conducted online, with devices and internet access being provided to members who did not have internet access. The survey was conducted in only English and Spanish. Many correction and bias-detection attempts were made, and the study mentions that they have various margins of error that are biased in favor and against certain criteria, such as being a grad/post-grad, having children in public school, etc.

	This survey acknowledges that the term ``STEM'' (science, technology, engineering, math) does not have a well-defined meaning, and certain standards organizations will provide conflicting definitions for STEM. For example, some entities consider medicine of any form to be part of STEM because of its roots in math, chemistry, and the scientific method.

	This survey %I say 'this survey' too much
defines STEM as the  U.S. Census Bureau’s American Community Survey defines it, and it includes about 74 distinct job titles.

	The findings from the Pew study show that women not only report more discrimination, 
but men percieve women as having less difficulty in the workplace as a result of their gender. 
This also applies to black interviewees as well.

	This trend of higher discrimination and lower percieved discrimination works together to enable
discrimination % find a different word... discrimination discrimination discrimination discrimination 
 while also downplaying the discrimination that occurs to workers.

	Every statistic from the percieved existence of a certain type of discrimination, to the reported occurrence
of said discrimination, points to non-white and female-gendered demographics being more affected than 
male-gendered and white demographics.

	At this time, without a different study that has taken place at a different time, no conclusion can
be made about how workplace discrimination with respect to gender and race in STEM has changed over time.

% okay, so find an article that goes over that!
	%I cannot use uhhhhhh forgot what i was gonna say fuck


\newpage
\singlespacing

\begin{thebibliography}{9}

\bibitem{csc}
``Ebony'' \textit{Ebony Magazine Vol. 35, No. 12, Page 61}\\
Johnson Publishing Company, \\
\url{https://books.google.com/books?id=3csDAAAAMBAJ&lpg=PA61&dq=war%20programmer&pg=PA61#v=onepage&q=war%20programmer&f=false}  \\
Accessed 9 Nov. 2019.

\bibitem{pew}
``Pew Research Center'' \textit{Women and Men in STEM Often at Odds Over Workplace Equity} \\
Pew Research Center,\\
\url{https://www.pewsocialtrends.org/2018/01/09/women-and-men-in-stem-often-at-odds-over-workplace-equity/}
Accessed 3 Dec. 2019.

\bibitem{punchcardpros}
``Ebony''
\textit{Ebony Magazine Vol. 24, No. 6, Page 113} \\
Johnson Publishing Company, \\
\url{https://books.google.com/books?id=HN8DAAAAMBAJ&lpg=PA113&dq=punch%20card&pg=PA113#v=onepage&q&f=false} \\
Accessed 3 Dec. 2019

\bibitem{congressrecord}
``Congressional Record: Proceedings and Debates of the 90th Congress: First Session''
\textit{U.S. GOVERNMENT HELPS LAUNCH TWO YOUNG PHILIDELPHIA NEGRO BUSINESSMEN ON A CAREER IN ELECTRONIC DATA PROCESSING} \\
United States Congress,
\url{https://books.google.com/books?id=BNZwcP4UuZEC&lpg=PA13940&ots=_z9FZw6R2i&dq=julius%20a%20baylor%20data%20preparation%20corp&pg=PA13940#v=onepage&q=julius%20a%20baylor%20data%20preparation%20corp&f=false} %TODO: use this
% this document calls Jack a negro. a sign of the times. use this source.

\end{thebibliography}

\newpage

\markdownInput{part-3-informal-notes.md}

\end{document}
