\documentclass[a4paper,12pt]{article}

\usepackage[utf8]{inputenc}
\usepackage{lipsum}  
\usepackage{url}
\usepackage[margin=1in]{geometry}

\usepackage{markdown}
\usepackage{setspace}

\title{HIST 380 Part 2: \\ `Breaking the loop' \\ Echoes of past inequality in high tech spaces, today }
\author{Henry Post, \url{hpost@iit.edu}}

\begin{document}

\maketitle

\newpage
\doublespacing

In the past, an oxymoron surrounded black women and minority groups in computing: They were both necessary but ignored and underpaid at the same time. Black women were necessary to World War 2, the space race, and supporting governmental financial systems when they worked as calculators, analysts, and mathematicians. The silence was retroactively covered up in some cases when all traces of their existance was willingly withheld.

Similar things can be said of the current state of diversity in technology: Many large companies are oblivious to or actively avoid addressing the problem of diversity in their work environments, especially at high levels of management. There have been various scandals and petitions against high tech companies for unequal treatment of women and people of color (POC) throughout the mid-2000s and especially in recent years.

While searching through Ebony Magazine, I was able to find a good amount of articles involving American companies that were supporting black employees in the workforce. The first one that I found was a Dec 1979 article\cite{csc} that advertised COBOL, FORTRAN, and PASCAL language positions as well as COMTEN, UNIVAC, and IBM 370 machine positions. This was for equal opportunity jobs under the Computer Sciences Corporation, a company who was acquired in 2017 by Hewlett-Packard.

I could find a few places in the London Times that involved women and punchcard operators, but none that were specifically pro-women, and especially none that were pro-black. There did not seem to be a vested interest in including women or black people in computing news or advertisements with the London Times.

I searched for NASA as well as NACA in the London times, but could not find anything other than job postings. The vast majority of computer and tech-related search results that I found in the Times were about female machine operators sought. One that struck me as rather odd was calling for young teenagers to work as machine operators\cite{youngmachineop}.

I was also trying to look for job postings or information specific to wartime or government jobs, but I could find government tax preparation job postings at best. There was one article that actually addresses the 1965 tax computer strikes \cite{taxstrikes} and talks about their escalation.

The opposite can be said of the articles regarding programming and technical jobs in Ebony magazine: There are tens of ads in the magazines that specifically ask for black and female operators. This actively includes black and female people in the tech space.

In fact, there was one black-owned company called Data Preparation Corp in Philidelphia, owned at the time by Julius A. (Jack) Baylor that dealt with data processing. This company was featured in an Ebony magazine article. The article details how Jack started with about \$20,000 and failed multiple times to start a company, but eventually succeeded and his company making about \$230,000 gross profit in the first two years.

From the sources I have available, the conclusion I must draw is that it seems that Britain did not make an effort to include women and people of color in its high tech spaces. In fact, Britain saw the negative effects of not including them as shown in their government having to deal with worker strikes over pay and discrimination, which was reported in the times. 

This is a direct contrast to what I could find for America from Ebony Magazine: It accepted and called for more black/female engineers and programmers.

% Is this unnecessary? Am I just crapping out my own beliefs? Eh, doesn't matter. It's 1AM!!!!! AAAAAAAAAAAAAAAAAAAAAAAAAAAAAAA!!!!!!

Just like in the past, many companies today have a vested interest in creating technologies that are divisive and exclusive. Racism and trolling generate attention, and for many platforms, attention directly corresponds with ad money. Policies implemented by companies are hamfisted and run by white dudes who really don't know what they're doing or what issues women, trans people, POC and other marginalized groups face, and how to fix those issues.

It doesn't have to be this way! There is a `speedbump' of compacency and dinosaur workplace culture that can cause difficulty for a company to overhaul the way that they think about diversity.

Through the power of cooperating with existing initiatives that promote women and POC in stem, grassroots initiatives led by employees, and cooperation and support from upper levels of management, companies like Google and Facebook can authentically support diversity in their workplace, and not just like they currently pretend to.

\newpage
\singlespacing

\begin{thebibliography}{9}

\bibitem{csc}
``Ebony'' \textit{Ebony Magazine Vol. 35, No. 12, Page 61}\\
Johnson Publishing Company, \\
\url{https://books.google.com/books?id=3csDAAAAMBAJ&lpg=PA61&dq=war%20programmer&pg=PA61#v=onepage&q=war%20programmer&f=false}  \\
Accessed 9 Nov.2019.

\bibitem{youngmachineop}
``Books And Prints." \textit{Times}, 7 July 1941, p.1. The Times Digital Archive, \\
\url{http://tinyurl.gale.com/tinyurl/CBqUj0}.\\
Accessed 10 Nov.2019.

\bibitem{taxstrikes}
``Computer Strikes To Be 'Escalated'." \textit{Times}, 18 June 1965, p.6. The Times Digital Archive,\\
\url{http://tinyurl.gale.com/tinyurl/CBq8D8}. \\
Accessed 10 Nov.2019.

\bibitem{punchcardpros}
``Ebony''
\textit{Ebony Magazine Vol. 24, No. 6, Page 113} \\
Johnson Publishing Company, \\
\url{https://books.google.com/books?id=HN8DAAAAMBAJ&lpg=PA113&dq=punch%20card&pg=PA113#v=onepage&q&f=false} \\
Accessed 11 Nov. 2019

\end{thebibliography}

\newpage

\markdownInput{part-2-informal-notes.md}

\end{document}
