\documentclass[a4paper,12pt]{article}

\usepackage[utf8]{inputenc}
\usepackage{lipsum}  
\usepackage{url}
\usepackage[margin=1in]{geometry}

\usepackage{markdown}

\title{HIST 380 Part 1: \\ IIT Tech News Archives \\ History of LGBTQIA people in \\ or adjacent to tech spaces at IIT}
\author{Henry Post, \url{hpost@iit.edu}, \url{henryfbp.me}}

\begin{document}

\maketitle

\newpage

	My focus for this part of the assignment was originally going to be foreign students, people of color (POC), and specifically black POC. I wanted to explore the unique relationship that IIT must have with race, racism, and gentrification given its location in Bronzeville, a historically black neighborhood. 

	However, I stumbled upon a few articles that were about homosexuality with wildly different dates, and found a very interesting series of sub-interactions between authors of the papers. I was able to find arguments for and against queer issues, and a very old (1970) article that was pro-gay rights.

	The first article \cite{antigay} I will examine is one written by Jonathan Mikesell for IIT Tech News in 2005. This article is a critique of pro-gay posters/advertisements, and gay culture as a whole. It rebukes the validity of gay struggles and gay rights, and it posits that homosexuality is not a part of identity because homosexuality is essentially a fetish for penises. The article goes on to reason that because homosexuality is not an intrinsic part of one's identity, homosexuals cannot be oppressed because it is a choice of lust to seek out sexual relations with other men.

	After examining the reason for gay people's attraction to oneanother, Mikesell posits that gays have not endured oppression\footnote{This is false. Historically, gays have been oppressed by the state, police, and society as a whole in many different ways across history. Google ``Stonewall'', ``DADT'', or ``Pink Triangle''.} because other groups, like black americans, had to endure jailings and beatings in the 1950s. However, gays are not harassed and are generally left alone. Instead of being content with equal rights, they must make their fetish everyone's business and demand more and more rights that they do not need as they are not oppressed.
	
	This article by Mikesell is obviously very inflammatory and partially incorrect in the base arguments it supports. Unsurprisingly, two counter-articles were immediately published in next week's publication by two different people.  The first counter-article to Mikesell's article is written by Nadim Haque\cite{counterantigay1}, a queer writer. 
	
	Haque first dismantles one of Mikesell's main points: the `myth' of gay oppression. Haque rightly accuses Mikesell of leaving out a mountain of historical gay oppression. Haque goes on to examine the argument of gayness as what is essentially a `penis fetish' as posited by Mikesell, and asks ``Is heretosexuality a fetish for a specific type of genitalia as well?''. Haque then asks, ``Why do gay problems\footnote{Marriage, domestic life} matter? How does it affect you?'' through a series of questions that address the role of marriages, domestic life, and private gay relationships. Haque argues that none of these will likely ever affect Mikesell in any significant way, therefore he should not care that gays are afforded those rights.
	
	A second author, Joseph Kaiser, also responds to Mikesell's article. \cite{counterantigay2} This rebuttal is much more sarcastic than the first one, but the main points of anti-history erasure. Kaiser sarcastically paints a hypothetical scenario where he is sharing an intimate moment with his female life partner of many years, and tells her that he only loves her because their genitals fit together. This is a satire of argument proposed by Mikesell that posits that gay relationships are the result of a genital fetish, and not love, affection, companionship, etc.

	This is a very interesting example of a straight guy who feels the need to push back against the LGBT+ movement as a whole (specifically gay men) essentially because it bothered him. Immediately after this, two separate authors, one LGBT+ and one straight, publish counter-articles. This interaction is a small cultural snapshot of the ways that different people with different ideas could have been interacting at the time. 
	
	I can say from my experience as a straight-presenting queer man with a lot of `bro' friends that a lot of straight guys and friends still think this way, and don't like LGBT issues being visible to them. I know a lot of guys like Mikesell!

	The last article that I could find that deals with gayness is one from 1970, written by Dave Daruszka.\cite{progayold} This article is slightly pro-gay, and definitely not anti-gay. It is a very short five-paragraph summary of the efforts of a group of gay students at University of Chicago who want to normalize gayness and not be seen as mentally ill or perverts, but normal people. 

\newpage

\begin{thebibliography}{9}

\bibitem{antigay}
Jonathan Mikesell.
\textit{Homosexuality imposing ideology?} \\
Illinois Institute of Technology, TechNews Volume 159, Chicago, IL, Tuesday, Oct. 4, 2005.
\url{http://216.47.157.203/technews/volume159/tnvol159no6.pdf#page=4}

\bibitem{counterantigay1}
Nadim Haque.
\textit{In response to Jonathan Mikesell's “Homosexuals imposing ideology?” article} \\
Illinois Institute of Technology, TechNews Volume 159, Chicago, IL, Tuesday, Oct. 11, 2005.
\url{http://216.47.157.203/technews/volume159/tnvol159no7.pdf#page=4}

\bibitem{counterantigay2}
Joseph Kaiser.
\textit{In response to Jonathan Mikesell's “Homosexuals imposing ideology?” article} \\
Illinois Institute of Technology, TechNews Volume 159, Chicago, IL, Tuesday, Oct. 11, 2005.
\url{http://216.47.157.203/technews/volume159/tnvol159no7.pdf#page=4}

\bibitem{progayold}
Dave Daruszka.
\textit{Homosexuals Seek Equality With Help of Gay Liberation} \\
Illinois Institute of Technology, TechNews Volume 159, Chicago, IL, Friday, Feb. 13, 1970.
\url{http://216.47.157.203/technews/volume88/tnvol88no2.pdf#page=3}


\end{thebibliography}

\newpage

\markdownInput{part-1-informal-notes.md}

\end{document}
